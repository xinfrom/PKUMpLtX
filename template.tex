%
% PKUMpLtX --- A LaTeX document class for 'Modern Physics Laboratory' in PKU based on `revtex4-2`
%
% Please read `README.md' and the template file before using
% 需要确保 font 选项指定的字体已安装! 具体参见 `README.md' 的说明.
\documentclass[font=default]{mpltx}

% 以下至 \begin{document} 都仅是本文件为了方便额外定义的命令, 写报告时不需要.
\hypersetup{colorlinks=true}% 超链接带颜色
\usepackage{xcolor}
\newcommand{\note}[1]{{\color{gray}#1}}
\NewDocumentCommand{\pkg}{s o m}{%
    \IfBooleanF{#1}{%
        \IfNoValueTF{#2}%
            {\href{https://www.ctan.org/pkg/#3}}%
            {\href{https://www.ctan.org/pkg/#2}}%
    }%
    {\textsf{#3}}%
}
\newcommand*\cs[1]{\texttt{\textbackslash #1}}
\newcommand*\env[1]{\textit{\texttt{#1}}}
\newcommand*\code[1]{\texttt{#1}}
\newcommand*\file[1]{\textbf{\texttt{#1}}}
\makeatletter
\newcommand\releasedate{%
    \href{https://github.com/CastleStar14654/PKUMpLtX/releases/tag/\mpltx@fileversion}%
        {\mpltx@filedate, \mpltx@fileversion}}
\makeatother
% 以上是本文件为了方便额外定义的命令, 写报告时不需要.
\usepackage{float}

\begin{document}

\title{体效应振荡器的工作特性和波导管的工作状态实验报告} % 切合报告内容, 简短明确, 可以不同于讲义
\author{李钰欣} % 这里 \emailphone 一定要紧跟在 \author 后方
\emailphone{2300011368@stu.pku.edu.cn}{(86)15816647600}
% 如果改用 \email 则仅需要邮箱参数
\affiliation{北京大学物理学院\quad 学号: 2300011368}
% % 可以使用 \zhdate 自动生成中文日期, 如
\date{\zhdate{2025/11/13}}
% % 也可使用 babel 的 \localedate, 如
% \date{\localedate{2020}{12}{1}}
% % 两者均会输出 `2020 年 12 月 1 日'
% 下面的 \date 的参数是为了自动输出正确版本号, 正式报告请替换为上面的两种 \date 之一
% \date{\releasedate}
\begin{abstract}
本实验通过观测体效应振荡器的工作特性、测量波导管的小驻波比、中驻波比和波导波长,
掌握了测量微波相关重要参量的办法,并对体效应振荡器的结构和工作原理及波导管的工作原理有了进一步的认识。

\end{abstract}
\keywords{体效应振荡器,微波,波导管}

\maketitle

\section{引言}
微波技术在科学研究中是一种重要的观测手段,其研究方法和测试设备都与无线电波不同。
本实验目的在于:学习微波的基本知识和基本测量技术,并以微波为科学研究手段来观测物理现象。
具体来讲,本实验通过观测体效应振荡器的工作特性来了解微波的产生,并通过波导管的使用来进一步掌握微波的性质。


 
\section{理论}\label{sec:theory}
体效应振荡器的工作基础是“耿效应”:在n型砷化镓单晶样品两端外加电场,会产生频率很高的电磁振荡,产生振荡的根本原因是
部分半导体材料中具有双能谷结构,当外加电场高到一定程度时时会导致高场畴产生和消失的周期性变化。

波导管有三种工作状态:匹配状态、驻波状态和混波状态。自由空间波长$\lambda$,相速度$v_g$,
波导波长$\lambda_g$,群速度$\mu$满足以下关系式:

$$
c = \lambda f
$$
$$
v_g = \lambda_g f
$$
$$
v_g \mu = c^2
$$

定义驻波比$\rho = \frac{{|E|}_{max}}{{|E|}_{min}}$,通过降低反射可以使驻波比变小,也更接近匹配状态。


\section{实验装置}
实验装置示意图:见图1

\begin{figure}[H]
  \centering
  \includegraphics[width=0.5\linewidth]{fig/instrument5.jpg}
  \caption{实验线路}
  \label{sec:instrument5}
\end{figure}

固态源提供微波信号,隔离器调节反射大小,衰减器调节微波振幅,单螺调配器调节驻波测量线终端位置。

实验开始前需预热信号源30分钟。

\section{结果及讨论}
一、体效应振荡器的工作特性\vspace{0.5cm}

测量0到13v时工作电流和8到13V的频率和相对功率(相对功率由检流计示数表示)。实验测得数据如下图所示:

\begin{figure}[H]
  \centering
  \includegraphics[width=1\linewidth]{fig/data5(1).png}
  \caption{体效应振荡器实验数据}
  \label{sec:data5(1)}
\end{figure}
\vspace{1cm}

二、观测波导管的工作状态\vspace{0.5cm}

1、测量波导管的小驻波比

调节单螺调配器直至波导管为最佳匹配状态,实验数据如下表

\begin{table}[!ht]
    \centering
    \caption{小驻波比实验数据}
    \begin{tabular}{|l|l|l|}
    \hline
测量次数         & $I_max$         & $I_min$              \\  \hline
1                 & 103               &87          \\ \hline
2                & 101.5              &87             \\ \hline
3                     & 103           &87              \\ \hline
4                     & 104           &87.5              \\ \hline
    \end{tabular}
\end{table}

检波晶体满足平方律,解得 $\rho$ = 1.087
\vspace{0.2cm}

2、测量波导管的中驻波比

实验数据如下表

\begin{table}[!ht]
    \centering
    \caption{中驻波比实验数据}
    \begin{tabular}{|l|l|l|}
    \hline
测量次数         & $I_max$         & $I_min$              \\  \hline
1                 & 103               &14          \\ \hline
    \end{tabular}
\end{table}

计算得$\rho$ = 2.712
\vspace{0.2cm}

3、测量波导管的波导波长

在频率为8.959GHz时,由检流计示数最小的左右两处得到的$x_1$、$x_2$取平均得到波节所在位置,邻近波节位置相减得到波导波长的一半。重复三次。

实验数据如下所示:

\begin{table}[!ht]
    \centering
    \caption{波导波长实验数据}
    \begin{tabular}{|l|l|l|l|l|l|l|l|}
    \hline
序号         & $x_1 /mm$     & $x_2 /mm$   &$x_1' /mm$    & $x_2' /mm$   & $x_min /mm$   & $x_min' /mm$    & $\frac{\lambda_{g}}{2} /mm$    \\  \hline
1            & 77.5          &81          &102.1         &105.5          & 79.25          &103.8             &24.55          \\ \hline
2            & 102.1          &105.5          &126.7         &130.1       & 103.8       &128.4          &24.6         \\ \hline
3            & 126.7          &130.1          &151.3         &154.6       & 128.4       &152.95          &24.55          \\ \hline
    \end{tabular}
\end{table}

计算得$\lambda_g$ = 49.12mm
\vspace{0.2cm}

4、测量驻波曲线从而求晶体检波率n

实验数据如下图:

\begin{figure}[H]
  \centering
  \includegraphics[width=1\linewidth]{fig/data5(2).png}
  \caption{晶体检波特性实验数据}
  \label{sec:data5(2)}
\end{figure}

由数据可知,电流为最大值一半时对应横轴宽度$\Delta l = 12.15mm$

代入公式$n = \frac{-0.3010}{lg cos \frac{\pi \Delta l}{\lambda_g}}$得,检波率为2.0485。



\section{结论}

本实验通过测量和计算得到体效应振荡管的工作特性曲线、波导管的小驻波比和中驻波比、波导管的波导波长、晶体检波率,
掌握了微波基本参量的测量方法和波导管的操作方法,并对体效应振荡器的结构和工作原理及波导管的工作原理有了更深入的了解。

\end{document}





