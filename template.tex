%
% PKUMpLtX --- A LaTeX document class for 'Modern Physics Laboratory' in PKU based on `revtex4-2`
%
% Please read `README.md' and the template file before using
% 需要确保 font 选项指定的字体已安装! 具体参见 `README.md' 的说明.
\documentclass[font=default]{mpltx}

% 以下至 \begin{document} 都仅是本文件为了方便额外定义的命令, 写报告时不需要.
\hypersetup{colorlinks=true}% 超链接带颜色
\usepackage{xcolor}
\newcommand{\note}[1]{{\color{gray}#1}}
\NewDocumentCommand{\pkg}{s o m}{%
    \IfBooleanF{#1}{%
        \IfNoValueTF{#2}%
            {\href{https://www.ctan.org/pkg/#3}}%
            {\href{https://www.ctan.org/pkg/#2}}%
    }%
    {\textsf{#3}}%
}
\newcommand*\cs[1]{\texttt{\textbackslash #1}}
\newcommand*\env[1]{\textit{\texttt{#1}}}
\newcommand*\code[1]{\texttt{#1}}
\newcommand*\file[1]{\textbf{\texttt{#1}}}
\makeatletter
\newcommand\releasedate{%
    \href{https://github.com/CastleStar14654/PKUMpLtX/releases/tag/\mpltx@fileversion}%
        {\mpltx@filedate, \mpltx@fileversion}}
\makeatother
% 以上是本文件为了方便额外定义的命令, 写报告时不需要.

\begin{document}

\title{晶体的电光效应} % 切合报告内容, 简短明确, 可以不同于讲义
\author{李钰欣} % 这里 \emailphone 一定要紧跟在 \author 后方
\emailphone{2300011368@stu.pku.edu.cn}{(86)15816647600}
% 如果改用 \email 则仅需要邮箱参数
\affiliation{北京大学物理学院\quad 学号: 2300011368}
% % 可以使用 \zhdate 自动生成中文日期, 如
% \date{\zhdate{2020/12/1}}
% % 也可使用 babel 的 \localedate, 如
% \date{\localedate{2020}{12}{1}}
% % 两者均会输出 `2020 年 12 月 1 日'
% 下面的 \date 的参数是为了自动输出正确版本号, 正式报告请替换为上面的两种 \date 之一
\date{\releasedate}
\begin{abstract}
  本实验通过控制两个偏振器和调节晶体两端电压来测量晶体半波电压,并通过相位补偿法测出云母样品的折射率差。
  实验初步探索了晶体的一次电光效应,对一次电光效应有了更深入的理解。
\end{abstract}
\keywords{晶体的一次电光效应, 双轴晶体的折射率差}

\maketitle

\section{引言}

晶体的电光效应是在现代光学和光电子技术中非常重要的物理效应,指的是某些晶体的折射率在外加电场的作用下发生改变的现象。
本实验主要研究晶体的一次电光效应,尝试测出晶体的半波电压值和电光系数,并用电光晶体作为相位补偿器,测量双折射样品的微小相位差。

 
\section{理论}\label{sec:theory}
当有外加静电场或低频场$E_0$存在时,介质的光学介电常量$\epsilon$是电场$E_0$的函数。对于足够小的$E_0$,$\epsilon$展开为

$\epsilon$ = $\epsilon^{(1)}$ + $\epsilon^{(2)}\cdot E$ + ...

在无反演对称性的介质里,电光效应是由与$E_0$成正比的比例系数为$\epsilon^{(2)}$的项所支配,如本实验所用的KD*P晶体。
KD*P晶体对称类型为$\overline{4}$2m晶类,电光系数矩阵为:

\begin{pmatrix}
 0 & 0 & 0\\
 0 & 0 & 0\\
 0 & 0 & 0\\
 r_{41} & 0 & 0\\
 0 & r_{41} & 0\\
 0 & 0 & r_{63}
\end{pmatrix}






\end{document}
