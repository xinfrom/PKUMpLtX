%
% PKUMpLtX --- A LaTeX document class for 'Modern Physics Laboratory' in PKU based on `revtex4-2`
%
% Please read `README.md' and the template file before using
% 需要确保 font 选项指定的字体已安装! 具体参见 `README.md' 的说明.
\documentclass[font=default]{mpltx}

% 以下至 \begin{document} 都仅是本文件为了方便额外定义的命令, 写报告时不需要.
\hypersetup{colorlinks=true}% 超链接带颜色
\usepackage{xcolor}
\newcommand{\note}[1]{{\color{gray}#1}}
\NewDocumentCommand{\pkg}{s o m}{%
    \IfBooleanF{#1}{%
        \IfNoValueTF{#2}%
            {\href{https://www.ctan.org/pkg/#3}}%
            {\href{https://www.ctan.org/pkg/#2}}%
    }%
    {\textsf{#3}}%
}
\newcommand*\cs[1]{\texttt{\textbackslash #1}}
\newcommand*\env[1]{\textit{\texttt{#1}}}
\newcommand*\code[1]{\texttt{#1}}
\newcommand*\file[1]{\textbf{\texttt{#1}}}
\makeatletter
\newcommand\releasedate{%
    \href{https://github.com/CastleStar14654/PKUMpLtX/releases/tag/\mpltx@fileversion}%
        {\mpltx@filedate, \mpltx@fileversion}}
\makeatother
% 以上是本文件为了方便额外定义的命令, 写报告时不需要.
\usepackage{float}

\begin{document}

\title{非线性对流斑图实验报告} % 切合报告内容, 简短明确, 可以不同于讲义
\author{李钰欣} % 这里 \emailphone 一定要紧跟在 \author 后方
\emailphone{2300011368@stu.pku.edu.cn}{(86)15816647600}
% 如果改用 \email 则仅需要邮箱参数
\affiliation{北京大学物理学院\quad 学号: 2300011368}
% % 可以使用 \zhdate 自动生成中文日期, 如
\date{\zhdate{2025/10/30}}
% % 也可使用 babel 的 \localedate, 如
% \date{\localedate{2020}{12}{1}}
% % 两者均会输出 `2020 年 12 月 1 日'
% 下面的 \date 的参数是为了自动输出正确版本号, 正式报告请替换为上面的两种 \date 之一
% \date{\releasedate}
\begin{abstract}
本实验通过学习扫描隧穿显微镜的操作和调试过程,并用STM来观测样品的表面形貌、得到原子分辨的图像,
来进一步理解其原理和结构及更深入的理解量子力学中的隧穿效应。

\end{abstract}
\keywords{STM,量子隧穿}

\maketitle

\section{引言}
STM以量子隧穿为基本原理,相较其他显微镜具有高分辨本领的优越性,且不需要粒子源也不需要透镜系统来聚焦。另外,
STM不仅能提供样品形貌的三维实空间信息,还能在微观尺度上对表面进行可控的加工并对所产生的纳米结构进行各种研究。
本实验用STM得到原子分辨的图像,从而对STM有更深入的理解。

 
\section{理论}\label{sec:theory}
在量子理论中,在V(r)>E的区域,薛定谔方程的解不一定为零,因此一个入射粒子要穿透一个V(r)>E的区域的概率是不为0的,
这就是隧穿效应。STM核心是当具有一定偏置电压的针尖与样品十分接近时,能观察到隧穿电流,通过隧穿电流的变化就可以得到样品表面形貌的信息。

STM通过控制xy方向的扫描电压来控制针尖位置,通过隧穿电流反馈到z向陶瓷来判断针尖与表面的距离。
STM分为两种工作模式:恒定电流模式和恒定高度模式,其中恒定高度模式只适宜测量小范围,小起伏表面。


\section{实验装置}
实验装置示意图:见图1

\begin{figure}[H]
  \centering
  \includegraphics[width=0.5\linewidth]{fig/instrument4.jpg}
  \caption{STM示意图}
  \label{sec:instrument4}
\end{figure}

实验装置主要由扫描家和电子控制单元构成,其中扫描架由两队陶瓷杆和一根陶瓷管支撑着的牢固结构组成。

正式开始前需要进行粗逼近:起点置0,x、y电压范围调至最大,选择自动进针直至警报,看z轴电压示数,
大于100V时逐步进针至100V再退针20步,小于100V时退针20步。

\section{结果及讨论}
最初设置偏压为1000mV,根据图像选择较平整区域,逐步缩小x、y电压范围来调整扫描范围,
其中扫描时间与扫描范围呈线性关系。

得到小范围的图像后,观察图像情况,若漂移明显或原子不清晰,可通过缩短扫描时间和降低偏压得到更清晰的图像,

设置扫描时间100ms,偏压1000mV,扫描范围6V×6V时得到图2:

\begin{figure}[H]
  \centering
  \includegraphics[width=0.5\linewidth]{fig/result4(1).png}
  \caption{6V×6V时图像}
  \label{sec:result4(1)}
\end{figure}

设置扫描时间50ms,偏压300mV,扫描范围4V×4V时得到图3:

\begin{figure}[H]
  \centering
  \includegraphics[width=0.5\linewidth]{fig/result4(2).png}
  \caption{4V×4V时图像}
  \label{sec:result4(2)}
\end{figure}

设置扫描时间50ms,偏压300mV,扫描范围2V×2V时得到图2:

\begin{figure}[H]
  \centering
  \includegraphics[width=0.5\linewidth]{fig/result4(3).png}
  \caption{2V×2V时图像}
  \label{sec:result4(3)}
\end{figure}

实验观察分析得到:

1.扫描时间越短,图像中斜线的倾斜角度越接近90度,说明抑制了漂移对成像的影响,但同时图像分辨率降低。

2.降低偏压,维持恒定电流,可以使得针尖距离样品更近,从而提高图片分辨率。
但是这样操作的前提条件是在大范围时寻找较平坦的部位放大,否则容易撞坏针尖。

3.通过提高恒定电流也可以提高分辨率。

4.HOPG的晶体结构在理论上是六角密排结构,其原子排列中没有中心原子,但在STM实验中观察到的是有心六角结构,
这是因为:

  STM成像原理:STM探测的是样品表面的电子态密度,而不是原子核的真实位置。

  石墨的电子结构:在石墨的每个六边形环中,存在两种不等价的碳原子位置(A位和B位),分别位于相邻两层中。
由于层间耦合和电子云分布的不对称性,B位碳原子的局域态密度较高,在STM图像中表现为更亮的点,看起来就像是六边形的中心有一个原子。




\section{结论}

本实验用STM得到原子分辨的图像,通过调节x、y电压范围来调节图像所属范围大小,
通过调节偏压、扫描时间和隧穿电流来调整图像分辨率,
在这个过程中对STM的工作原理有了更深入的认识,也对STM的使用方法有了更进一步的了解。

\end{document}





