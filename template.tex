%
% PKUMpLtX --- A LaTeX document class for 'Modern Physics Laboratory' in PKU based on `revtex4-2`
%
% Please read `README.md' and the template file before using
% 需要确保 font 选项指定的字体已安装! 具体参见 `README.md' 的说明.
\documentclass[font=default]{mpltx}

% 以下至 \begin{document} 都仅是本文件为了方便额外定义的命令, 写报告时不需要.
\hypersetup{colorlinks=true}% 超链接带颜色
\usepackage{xcolor}
\newcommand{\note}[1]{{\color{gray}#1}}
\NewDocumentCommand{\pkg}{s o m}{%
    \IfBooleanF{#1}{%
        \IfNoValueTF{#2}%
            {\href{https://www.ctan.org/pkg/#3}}%
            {\href{https://www.ctan.org/pkg/#2}}%
    }%
    {\textsf{#3}}%
}
\newcommand*\cs[1]{\texttt{\textbackslash #1}}
\newcommand*\env[1]{\textit{\texttt{#1}}}
\newcommand*\code[1]{\texttt{#1}}
\newcommand*\file[1]{\textbf{\texttt{#1}}}
\makeatletter
\newcommand\releasedate{%
    \href{https://github.com/CastleStar14654/PKUMpLtX/releases/tag/\mpltx@fileversion}%
        {\mpltx@filedate, \mpltx@fileversion}}
\makeatother
% 以上是本文件为了方便额外定义的命令, 写报告时不需要.

\begin{document}

\title{非线性对流斑图实验报告} % 切合报告内容, 简短明确, 可以不同于讲义
\author{李钰欣} % 这里 \emailphone 一定要紧跟在 \author 后方
\emailphone{2300011368@stu.pku.edu.cn}{(86)15816647600}
% 如果改用 \email 则仅需要邮箱参数
\affiliation{北京大学物理学院\quad 学号: 2300011368}
% % 可以使用 \zhdate 自动生成中文日期, 如
\date{\zhdate{2025/10/16}}
% % 也可使用 babel 的 \localedate, 如
% \date{\localedate{2020}{12}{1}}
% % 两者均会输出 `2020 年 12 月 1 日'
% 下面的 \date 的参数是为了自动输出正确版本号, 正式报告请替换为上面的两种 \date 之一
% \date{\releasedate}
\begin{abstract}
本实验通过设置电流调整上下表面温差,并利用阴影法,得到不同温差、不同流体厚度下的对流斑图。
通过本实验,对对流现象有了更深入的认识,也对耗散结构理论有了初步的了解。

\end{abstract}
\keywords{非线性热对流,耗散结构}

\maketitle

\section{引言}

因热力作用使得流体状态(密度、压力等)发生差异可以引起流体运动,称为热对流。
对具有自由面-固壁底层的流体薄层进行底层加热,可以观察到各种对流图形,称为斑图。
本实验观测非线性热对流斑图,从而对耗散结构理论有一个初步的认识。

 
\section{理论}\label{sec:theory}
由NS方程、流体连续性方程、热传导方程,并假定存在微扰$T = T_0(z) + \theta$,$p = p_0(z) + p'$,$\mathbf{V} = \mathbf{u}$,
代入方程组并作去量纲化可以得到:

$$
\begin{cases}
\sigma^{-1} (\frac{\partial u}{\partial t} + \mathbf{u} \cdot \nabla \mathbf{u}) = \theta z - \nabla \mathbf{p} + \nabla^2 \mathbf{u} \\
\nabla \cdot \mathbf{u} = 0 \\
\frac{\partial \theta}{\partial t} + \mathbf{u} \cdot \nabla \theta = Ra \omega + \nabla^2 \theta
\end{cases}
$$

其中边界条件为$\theta = u = 0$, $z = \pm \frac{1}{2}$,$Ra \equiv \frac{g \alpha d^3 \Delta T}{\kappa \gamma}$
被称为瑞利数。

通过数值计算结果显示有临界参量:$R_c = 1707.76, q_c = 3.117$,也就是说,当对流水层的上下温差使得瑞利数Ra<$R_c$时,系统保持定态解稳定;
当Ra>$R_c$时,满足条件的部分扰动噪声就会逐渐变大,大到一定程度时,方程中的非线性项就不能忽略,进而得到的解可以理解实验结果的斑图分叉行为。


\section{实验装置}
实验装置示意图:见图1

\begin{figure}
  \centering
  \includegraphics[width=0.85\linewidth]{fig/instrument3.jpg}
  \caption{非线性对流斑图实验装置示意图}
  \label{sec:instrument3}
\end{figure}

主要装置:

小薄层的对流水层,其上为降温水层,两个水层间为蓝宝石片(热导率46W/m $\cdot$ K);对流水层下表面是黄金镀黄铜盘,利用硅胶片通电加热保持其温度均匀性。

实验装置中光路部分得到准平行光光束,进而通过阴影法得到斑图。

\section{结果及讨论}
1.实验数据
水层为2mm时:

\begin{table}[!ht]
    \centering
    \begin{tabular}{|l|l|l|l|}
    \hline
电流(mA)         & 下层温度(摄氏度)         &上层温度(摄氏度)        &温差(摄氏度)         \\  \hline
0                     & 24.4                      &26.1              &-1.7 \\ \hline
500                    & 26.9                      &26.9             &0      \\ \hline
850                     & 32.3                     &28.7              &3.6      \\ \hline
900                     & 34.7                    &29.8               &4.9  \\ \hline
950                      &36.3                     &30.9            & 5.4   \\ \hline
1000                     & 37.9                    &31.9              &6       \\ \hline
1200                    & 42.3                     &34.0              &8.3      \\ \hline
    \end{tabular}
\end{table}

水层为4mm时:

\begin{table}[!ht]
    \centering
    \begin{tabular}{|l|l|l|l|}
    \hline
电流(mA)         & 下层温度(摄氏度)       &上层温度(摄氏度)     &温差(摄氏度)         \\  \hline
0                     & 27.6                      &29.7              &-2.1     \\ \hline
200                    & 27.3                      &29.3             &-2       \\ \hline
400                     & 28.8                    &29.3              &-0.5     \\ \hline
600                     & 31.1                    &29.4              &1.7      \\ \hline
900                      &36.2                     &30.3            & 5.9      \\ \hline
1200                     & 43.6                    &32.0              &11.6    \\ \hline
1500                    & 42.3                     &34.0              &8.3     \\ \hline
    \end{tabular}
\end{table}

2.现象及分析


1.能量刻度

使用$ ^{137}{Cs}$测量0.662MeV光电峰位置及各参量,使用$ ^{60}{Co}$测量1.17MeV和1.33MeV光电峰位置及各参量,其中测量时间设为10min

测量结果:

\begin{table}[!ht]
    \centering
    \begin{tabular}{|l|l|}
    \hline
能量(MeV)                & 道址                  \\  \hline
0.662                     & 459                    \\ \hline
1.17                      & 812                    \\ \hline
1.33                      & 916                     \\ \hline
    \end{tabular}
\end{table}

数据处理:见图2 

\begin{figure}
  \centering
  \includegraphics[width=0.85\linewidth]{fig/data2(1).png}
  \caption{能量刻度}
  \label{sec:data2(1)}
\end{figure}

即$E = {10}^{-3} \times (1.4559n - 7.3754)$,单位为MeV。

2.测量不同散射角时的散射光子能谱以及本底

插上散射铝棒,记录不同散射角时的道址,左右道位置和重点区面积。拔出散射铝棒,测量各角度对应相同道数区间的本底面积。

测量数据如表:

\begin{table}[!ht]
    \centering
    \begin{tabular}{|l|l|l|l|l|l|l|}
    \hline
$\theta$(°)                & 道址                 &左道址            &右道址            &总面积     &本底面积    &净面积 \\ \hline
20                   & 421                  &389                &447            &14371     &846        &13525 \\ \hline
40                   & 346                   &320                &370           &9982      &417        &9565 \\ \hline
60                   & 266                   &245                &297           &8800      &421        &8379  \\ \hline
80                   & 219                  &191                  &239          &8169      &500        &7669  \\ \hline
100                  & 174                  &151                 &193           &8530      &615        &7915  \\ \hline
120                  &145                   &129                 &161           &9128      &709        &8419  \\ \hline
    \end{tabular}
\end{table}

数据处理:

微分散射截面$\frac{d \sigma (\theta)}{d \Omega} = \frac{N_p(\theta)}{R(\theta)\eta(\theta)4 \pi N_0 N_e f}$,其中$N_p(\theta)$对应上表中净面积,
因此需求出$R(\theta)$和$\eta(\theta)$,之后才可求解相对散射截面。本实验设定$\theta_0$ = 20°。

根据课本数据,用最小二乘法得到:$R(\theta) =-0.71828 E(\theta) + 0.918046$,
$\eta(\theta) = 10^{-3} \times (-0.62907E(\theta) + 11.12064)$,其中E单位为MeV。

计算得各角度对应$R$,$\eta$,相对散射截面如表:

\begin{table}[!ht]
    \centering
    \begin{tabular}{|l|l|l|l|l|l|l|}
    \hline
$\theta$(°)         & 20                     &40                             &60                   &80                     &100                        &120 \\ \hline
E(keV)              & 605.559               &496.366                        &379.894                 &311.466              &245.951                 &203.730 \\ \hline
$R(\theta)$          & 0.4831                 &0.5615                           &0.6451              &0.6943                 &0.7413                    &0.7717 \\ \hline
$\eta(\theta)$     & $7.40 \times {10}^{-4}$  & $8.08 \times {10}^{-4}$  &$8.82 \times {10}^{-4}$ &$9.24 \times {10}^{-4}$  &$9.66 \times {10}^{-4}$     &$9.92 \times {10}^{-4}$    \\ \hline
相对微分截面          & 1                     &0.556729                         &0.389184               &0.315584            &0.292018                     &0.290421  \\ \hline
    \end{tabular}
\end{table}

与课本数据对比:见图3,图4

\begin{figure}
  \centering
  \includegraphics[width=0.85\linewidth]{fig/figure_1.png}
  \caption{E- $\theta$}
  \label{sec:figure_1}
\end{figure}

\begin{figure}
  \centering
  \includegraphics[width=0.85\linewidth]{fig/figure_2.png}
  \caption{$\frac{d \sigma (\theta)}{d \Omega}$-$\theta$}
  \label{sec:figure_2}
\end{figure}

分析实验数据可以看到,实验所测能量略低于课本表格的能量,微分散射截面基本大于课本数据。

误差分析:

1.实验条件下,测量数据不对应某个准确的立体角,而对应该立体角附近一定展宽,因此微分散射截面偏大

2.在测量重点区面积时,由于仪器分辨率、测量时间等原因,在一定的道址范围内都能满足道计数为峰值约三分之一的标准,因此测量重点区面积会有误差,对微分散射截面测量带来误差。



\section{结论}
 本实验通过测量经过康普顿散射后光子的能量和相对微分散射截面,验证了康普顿散射效应,即散射后光子的能量会随角度变化,散射后光子的微分截面会随角度变化。
 通过本实验也对康普顿效应有了进一步的了解。

\end{document}





