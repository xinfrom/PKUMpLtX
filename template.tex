%
% PKUMpLtX --- A LaTeX document class for 'Modern Physics Laboratory' in PKU based on `revtex4-2`
%
% Please read `README.md' and the template file before using
% 需要确保 font 选项指定的字体已安装! 具体参见 `README.md' 的说明.
\documentclass[font=default]{mpltx}

% 以下至 \begin{document} 都仅是本文件为了方便额外定义的命令, 写报告时不需要.
\hypersetup{colorlinks=true}% 超链接带颜色
\usepackage{xcolor}
\newcommand{\note}[1]{{\color{gray}#1}}
\NewDocumentCommand{\pkg}{s o m}{%
    \IfBooleanF{#1}{%
        \IfNoValueTF{#2}%
            {\href{https://www.ctan.org/pkg/#3}}%
            {\href{https://www.ctan.org/pkg/#2}}%
    }%
    {\textsf{#3}}%
}
\newcommand*\cs[1]{\texttt{\textbackslash #1}}
\newcommand*\env[1]{\textit{\texttt{#1}}}
\newcommand*\code[1]{\texttt{#1}}
\newcommand*\file[1]{\textbf{\texttt{#1}}}
\makeatletter
\newcommand\releasedate{%
    \href{https://github.com/CastleStar14654/PKUMpLtX/releases/tag/\mpltx@fileversion}%
        {\mpltx@filedate, \mpltx@fileversion}}
\makeatother
% 以上是本文件为了方便额外定义的命令, 写报告时不需要.
\usepackage{float}

\begin{document}

\title{扫描隧穿显微镜实验报告} % 切合报告内容, 简短明确, 可以不同于讲义
\author{李钰欣} % 这里 \emailphone 一定要紧跟在 \author 后方
\emailphone{2300011368@stu.pku.edu.cn}{(86)15816647600}
% 如果改用 \email 则仅需要邮箱参数
\affiliation{北京大学物理学院\quad 学号: 2300011368}
% % 可以使用 \zhdate 自动生成中文日期, 如
\date{\zhdate{2025/11/13}}
% % 也可使用 babel 的 \localedate, 如
% \date{\localedate{2020}{12}{1}}
% % 两者均会输出 `2020 年 12 月 1 日'
% 下面的 \date 的参数是为了自动输出正确版本号, 正式报告请替换为上面的两种 \date 之一
% \date{\releasedate}
\begin{abstract}
本实验通过观测体效应振荡器的工作特性、测量波导管的小驻波比、中驻波比和波导波长,
掌握了测量微波相关重要参量的办法,并对体效应振荡器的结构和工作原理及波导管的工作原理有了进一步的认识。

\end{abstract}
\keywords{体效应振荡器,微波,波导管}

\maketitle

\section{引言}
微波技术在科学研究中是一种重要的观测手段,其研究方法和测试设备都与无线电波不同。
本实验目的在于:学习微波的基本知识和基本测量技术,并以微波为科学研究手段来观测物理现象。
具体来讲,本实验通过观测体效应振荡器的工作特性来了解微波的产生,并通过波导管的使用来进一步掌握微波的性质。


 
\section{理论}\label{sec:theory}
体效应振荡器的工作基础是“耿效应”:在n型砷化镓单晶样品两端外加电场,会产生频率很高的电磁振荡,产生振荡的根本原因是
部分半导体材料中具有双能谷结构,当外加电场高到一定程度时时会导致高场畴产生和消失的周期性变化。

波导管有三种工作状态:匹配状态、驻波状态和混波状态。自由空间波长$\lambda$,相速度$v_g$,
波导波长$\lambda_g$,群速度$\mu$满足以下关系式:

$$
c = \lambda f
v_g = \lambda_g f
v_g \mu = c^2
$$

定义驻波比$\rho = \frac{{|E|}_max}{{|E|}_min}$,通过降低反射可以使驻波比变小,也更接近匹配状态。


\section{实验装置}
实验装置示意图:见图1

\begin{figure}[H]
  \centering
  \includegraphics[width=0.5\linewidth]{fig/instrument5.jpg}
  \caption{实验线路}
  \label{sec:instrument5}
\end{figure}

固态源提供微波信号,隔离器调节反射大小,衰减器调节微波振幅,单螺调配器调节驻波测量线终端位置。

实验开始前需预热信号源30分钟。

\section{结果及讨论}
一、体效应振荡器的工作特性\vspace{1cm}

测量0到13v时工作电流和8到13V的频率和相对功率(相对功率由检流计示数表示)。实验测得数据如下图所示:

\begin{figure}[H]
  \centering
  \includegraphics[width=0.5\linewidth]{fig/data5(1).png}
  \caption{体效应振荡器实验数据}
  \label{sec:data5(1)}
\end{figure}
\vspace{2cm}

二、观测波导管的工作状态\vspace{1cm}





最初设置偏压为1000mV,根据图像选择较平整区域,逐步缩小x、y电压范围来调整扫描范围,
其中扫描时间与扫描范围呈线性关系。

得到小范围的图像后,观察图像情况,若漂移明显或原子不清晰,可通过缩短扫描时间和降低偏压得到更清晰的图像,

设置扫描时间100ms,偏压1000mV,扫描范围6V×6V时得到图2:

\begin{figure}[H]
  \centering
  \includegraphics[width=0.5\linewidth]{fig/result4(1).png}
  \caption{6V×6V时图像}
  \label{sec:result4(1)}
\end{figure}

设置扫描时间50ms,偏压300mV,扫描范围4V×4V时得到图3:

\begin{figure}[H]
  \centering
  \includegraphics[width=0.5\linewidth]{fig/result4(2).png}
  \caption{4V×4V时图像}
  \label{sec:result4(2)}
\end{figure}

设置扫描时间50ms,偏压300mV,扫描范围2V×2V时得到图4:

\begin{figure}[H]
  \centering
  \includegraphics[width=0.5\linewidth]{fig/result4(3).png}
  \caption{2V×2V时图像}
  \label{sec:result4(3)}
\end{figure}

实验观察分析得到:

1.扫描时间越短,图像中斜线的倾斜角度越接近90度,说明抑制了漂移对成像的影响,但同时图像分辨率降低。

2.降低偏压,维持恒定电流,可以使得针尖距离样品更近,从而提高图片分辨率。
但是这样操作的前提条件是在大范围时寻找较平坦的部位放大,否则容易撞坏针尖。

3.通过提高恒定电流也可以提高分辨率。

4.HOPG的晶体结构在理论上是六角密排结构,其原子排列中没有中心原子,但在STM实验中观察到的是有心六角结构,
这是因为:

  STM成像原理:STM探测的是样品表面的电子态密度,而不是原子核的真实位置。

  石墨的电子结构:在石墨的每个六边形环中,存在两种不等价的碳原子位置(A位和B位),分别位于相邻两层中。
由于层间耦合和电子云分布的不对称性,B位碳原子的局域态密度较高,在STM图像中表现为更亮的点,看起来就像是六边形的中心有一个原子。



\section{结论}

本实验用STM得到原子分辨的图像,通过调节x、y电压范围来调节图像所属范围大小,
通过调节偏压、扫描时间和隧穿电流来调整图像分辨率,
在这个过程中对STM的工作原理有了更深入的认识,也对STM的使用方法有了更进一步的了解。

\end{document}





