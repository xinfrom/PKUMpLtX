%
% PKUMpLtX --- A LaTeX document class for 'Modern Physics Laboratory' in PKU based on `revtex4-2`
%
% Please read `README.md' and the template file before using
% 需要确保 font 选项指定的字体已安装! 具体参见 `README.md' 的说明.
\documentclass[font=default]{mpltx}

% 以下至 \begin{document} 都仅是本文件为了方便额外定义的命令, 写报告时不需要.
\hypersetup{colorlinks=true}% 超链接带颜色
\usepackage{xcolor}
\newcommand{\note}[1]{{\color{gray}#1}}
\NewDocumentCommand{\pkg}{s o m}{%
    \IfBooleanF{#1}{%
        \IfNoValueTF{#2}%
            {\href{https://www.ctan.org/pkg/#3}}%
            {\href{https://www.ctan.org/pkg/#2}}%
    }%
    {\textsf{#3}}%
}
\newcommand*\cs[1]{\texttt{\textbackslash #1}}
\newcommand*\env[1]{\textit{\texttt{#1}}}
\newcommand*\code[1]{\texttt{#1}}
\newcommand*\file[1]{\textbf{\texttt{#1}}}
\makeatletter
\newcommand\releasedate{%
    \href{https://github.com/CastleStar14654/PKUMpLtX/releases/tag/\mpltx@fileversion}%
        {\mpltx@filedate, \mpltx@fileversion}}
\makeatother
% 以上是本文件为了方便额外定义的命令, 写报告时不需要.
\usepackage{float}

\begin{document}

\title{高压强真空电离计的校准实验报告} % 切合报告内容, 简短明确, 可以不同于讲义
\author{李钰欣} % 这里 \emailphone 一定要紧跟在 \author 后方
\emailphone{2300011368@stu.pku.edu.cn}{(86)15816647600}
% 如果改用 \email 则仅需要邮箱参数
\affiliation{北京大学物理学院\quad 学号: 2300011368}
% % 可以使用 \zhdate 自动生成中文日期, 如
\date{\zhdate{2025/12/11}}
% % 也可使用 babel 的 \localedate, 如
% \date{\localedate{2020}{12}{1}}
% % 两者均会输出 `2020 年 12 月 1 日'
% 下面的 \date 的参数是为了自动输出正确版本号, 正式报告请替换为上面的两种 \date 之一
% \date{\releasedate}
\begin{abstract}
本实验通过膨胀法校准真空计,用一级膨胀系统在$10^-3 ~1$Torr范围内进行校准。
经过本实验,对真空计的使用有了一定了解,也初步掌握了使用机械泵和扩散泵抽真空的办法和用膨胀法测量真空度的方法,
并进一步加深了对相关原理的理解。


\end{abstract}
\keywords{高压强真空电离计,电离真空计的校准,膨胀法}

\maketitle

\section{引言}
超导电性是低温物理的一个重要研究领域,从1911年首次发现超导电性现象以来,这一研究领域已诞生至少10位诺贝尔物理奖
获得者。另外,1962年约瑟夫森预言了两块超导体被一薄绝缘层分开时库伯对的量子隧穿现象,即约瑟夫森效应。
目前约瑟夫森效应已在灵敏探测、电压基准及量子线路等领域获得广泛应用。


 
\section{理论}\label{sec:theory}
高压强电离规的原理基于气体分子在强电场和磁场中被电离。
其核心是:在放电空间中,气体分子被高能电子碰撞电离,产生的离子流与气体压强在一定范围内成正比,
通过测量这个离子流的大小,就可以推算出气体的压强。

当压强升高到电子平均自由程小于电极间电子的平均路程时,平均每个电子有可能碰撞两个或更多的气体分子,线性关系不再成立。
DL-8型高压强电离真空计通过改变电极结构使得电子路程缩短,从而扩大压强测量的量程。

膨胀法是校准真空计的一种基本方法,其工作原理的基础是理想气体定律:恒温下pV=常量。
膨胀法校准系统的原理示意图见图1。整个系统先用机械泵﹣扩散泵系统抽至高真空,关闭活塞B、C和D,
开通活塞A充入适量的干燥空气,使稳压瓶内压强p。可由左边U形管真空计的读数$\delta h$准确测出(一般为几个托).
打开活塞B,随即再将它关闭,由于B到C之间的体积V。很小,这时Vo中的压强就是$p_0$。
然后打开活塞C使$V_0$中的气体膨胀到C和D之间的大体积$V_1$中。
由于 $V_1$原已抽至高真空,其本底压强可以忽略。设膨胀后气体压强为$p_1$,则:

${p_1}^{(1)} = \frac{V_0}{V_0 + V_1} p_0$

${p_1}^{(n)} = \frac{p_0 V_0 + {p_1}^{(n-1)} V_1}{V_0 + V_1}$

\begin{figure}[H]
  \centering
  \includegraphics[width=0.5\linewidth]{fig/theory7.jpg}
  \caption{膨胀法真空校准系统原理图}
  \label{sec:theory7}
\end{figure}


\section{实验装置}
实验装置示意图:见图2

\begin{figure}[H]
  \centering
  \includegraphics[width=0.5\linewidth]{fig/instrument7.jpg}
  \caption{膨胀法真空校准系统实验装置示意图}
  \label{sec:instrument7}
\end{figure}



\section{结果及讨论}
一、DL-8电离工作计正常工作状态的确认

电离计处于${10}^{-1}$档时,万用表接收集极0和辅助极8时,发现示数为0,因此用开关连接辅助极与阴极。
同理,电离计处于${10}^{-3}$档时,用开关连接辅助极与阳极。

各档位测得电压如下表:

\begin{table}[!ht]
    \centering
    \caption{各档位各极电压}
    \begin{tabular}{|l|l|l|l|}
    \hline
档位                  & 阳极               & 阴极               &辅助极             \\  \hline
${10}^{0}$            & 167.1              &50.4               &50.9         \\ \hline
${10}^{-1}$           & 167.4              &63.5               &62.4             \\ \hline
${10}^{-2}$           & 162.1              &50.3               &50.9             \\ \hline
${10}^{-3}$           & 167.1              &9.78               &166.8              \\ \hline
    \end{tabular}
\end{table}

二、压强测量实验结果

测量结果如图3所示,其中横轴代表测量并计算得到的压强的自然对数,纵轴代表真空计示数的对数。



可以看到实验数据大致满足y=x的关系,分区间看会发现随着压强增大,斜率逐渐小于1。这说明压强变大时,计算得到的压强偏大

原因分析:1、压强较低时每次取出的气体较少,p_0基本不变,压强较高时p_0的改变相较较大,不能当成常量迭代。
2、实验仪器存在漏气现象,压强越大漏气越严重。

因此最高工作压强约为${10}^{0}$Torr。



\section{结论}
本次实验通过一级膨胀法测量压强并对真空计进行校准。通过本实验,对真空的制造和真空度的测量技术有了更深入的
认识,也初步掌握了真空计的使用方法和校准方法。另外,通过对实验曲线的分析,可以看到实验中漏气、$p_0$的变化等
是如何影响测量结果的。

\vspace{1.5cm}

最后附上实验记录如图:

\begin{figure}[H]
  \centering
  \includegraphics[width=1\linewidth]{fig/note7(1).jpg}
  \label{sec:note7(1)}
\end{figure}

\begin{figure}[H]
  \centering
  \includegraphics[width=1\linewidth]{fig/note7(2).jpg}
  \label{sec:note7(2)}
\end{figure}

\begin{figure}[H]
  \centering
  \includegraphics[width=1\linewidth]{fig/note7(3).jpg}
  \label{sec:note7(3)}
\end{figure}

\begin{figure}[H]
  \centering
  \includegraphics[width=1\linewidth]{fig/note7(4).jpg}
  \label{sec:note7(4)}
\end{figure}


\end{document}





