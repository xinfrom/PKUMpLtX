%
% PKUMpLtX --- A LaTeX document class for 'Modern Physics Laboratory' in PKU based on `revtex4-2`
%
% Please read `README.md' and the template file before using
% 需要确保 font 选项指定的字体已安装! 具体参见 `README.md' 的说明.
\documentclass[font=default]{mpltx}

% 以下至 \begin{document} 都仅是本文件为了方便额外定义的命令, 写报告时不需要.
\hypersetup{colorlinks=true}% 超链接带颜色
\usepackage{xcolor}
\newcommand{\note}[1]{{\color{gray}#1}}
\NewDocumentCommand{\pkg}{s o m}{%
    \IfBooleanF{#1}{%
        \IfNoValueTF{#2}%
            {\href{https://www.ctan.org/pkg/#3}}%
            {\href{https://www.ctan.org/pkg/#2}}%
    }%
    {\textsf{#3}}%
}
\newcommand*\cs[1]{\texttt{\textbackslash #1}}
\newcommand*\env[1]{\textit{\texttt{#1}}}
\newcommand*\code[1]{\texttt{#1}}
\newcommand*\file[1]{\textbf{\texttt{#1}}}
\makeatletter
\newcommand\releasedate{%
    \href{https://github.com/CastleStar14654/PKUMpLtX/releases/tag/\mpltx@fileversion}%
        {\mpltx@filedate, \mpltx@fileversion}}
\makeatother
% 以上是本文件为了方便额外定义的命令, 写报告时不需要.
\usepackage{float}

\begin{document}

\title{饱和吸收光谱实验报告} % 切合报告内容, 简短明确, 可以不同于讲义
\author{李钰欣} % 这里 \emailphone 一定要紧跟在 \author 后方
\emailphone{2300011368@stu.pku.edu.cn}{(86)15816647600}
% 如果改用 \email 则仅需要邮箱参数
\affiliation{北京大学物理学院\quad 学号: 2300011368}
% % 可以使用 \zhdate 自动生成中文日期, 如
\date{\zhdate{2026/1/20}}
% % 也可使用 babel 的 \localedate, 如
% \date{\localedate{2020}{12}{1}}
% % 两者均会输出 `2020 年 12 月 1 日'
% 下面的 \date 的参数是为了自动输出正确版本号, 正式报告请替换为上面的两种 \date 之一
% \date{\releasedate}

% \begin{abstract}



% \end{abstract}
% \keywords{高压强真空电离计,电离真空计的校准,膨胀法}

\maketitle

\section{引言}
饱和吸收光谱是一种常用的精密激光光谱技术,其原理是利用单色可调谐激光,
将速度为0的原子从其多普勒速度分布的背景原子气体中选出,并使其对探测激光的吸收产生
饱和,形成饱和吸收光谱。


\section{理论}\label{sec:theory}
对于静止原子,当入射光频率等于原子某两能级间的跃迁频率时会发生共振吸收,进而得到吸收光谱和荧光光谱。
但当中心频率为$\omega _0$的原子以速度v运动时,跃迁频率为$\omega _d = \omega _0 (1$ ± $ \frac{v}{c})$,
由于源自速度分布很宽,最终会产生多普勒吸收/荧光谱。

饱和吸收法是一种用来消除多普勒本底的办法。假设原子具有三能级,在原子气室中对射两束相同频率的激光:
泵浦光$I_1$与探测光$I_2$,只有满足$\omega_{12} = \omega$/$\omega_{13} = \omega$的静止原子
或z向速度v满足$\omega + kv = \omega_{13} / \omega_{12},\omega - kv = \omega_{12} / \omega_{13}$,
才能同时吸收泵浦光和探测光,发生不同能级间的共振跃迁,其中前者形成静止原子饱和吸收光谱,后者形成交叉饱和吸收光谱。 




\section{实验装置}
实验装置示意图:见图1

\begin{figure}[H]
  \centering
  \includegraphics[width=0.5\linewidth]{fig/instrument8.jpg}
  \caption{饱和吸收实验装置示意图}
  \label{sec:instrument8}
\end{figure}

搭建光路后,用光电探测器接收穿过原子样品之后的探测光,并将其输出到示波器上
进行观察。扫描激光频率,调节示波器各旋钮,即可在示波器上观测到饱和吸收谱。

\section{结果及讨论}
可以在示波器上观察到在大的波形(即多普勒吸收峰)的波谷处有数个小的尖锐的波峰,即为饱和吸收峰。

可以总结出,饱和吸收峰的产生,就是当激光频率扫描至某一特定频率时,有一特定速度群的原子能够同时与泵浦光和探测光发生共振跃迁,
由于泵浦光比较强,几乎消耗了所有该速度群的基态原子,造成原子不再吸收探测光而呈现出透明状态(也即饱和,在光谱上看到是探测光强增大),
使得较弱的探测光得以通过气室形成透射峰,而当激光频率偏离这一特定频率时,泵浦光和探测光分别被不同速度群的原子吸收,形成多普勒吸收本底,
从而得到饱和吸收光谱。



\section{结论}
本实验通过饱和吸收法观察得到饱和吸收峰,若进一步测量可以得到铷原子发生跃迁的本征频率。在此过程中对光路的调节、搭建技巧有了进一步的认识,
也通过实验对相关理论有了进一步了解。


\section{感想}
首先是做实验切忌没想清楚就上手。以光路搭建为例,本实验用到的只是简单的几何知识(反射透射),应想清楚如何按顺序调好每一个元件(可以用简单的几个点来确定),
而非来回调节多个元件。
其次是,做实验需要耐心,尤其像光路的搭建,需要每一步都尽可能将光路调到最优的状态,否则很可能导致实验现象不明显而不得不重新搭建光路。总之切忌贪快。
第三,得到结果后实验并非就此结束了,需要继续尝试调试光路等,以期优化得到更好的实验结果,不能像本科课程一样抱着一种“完成就好”的心态,而应更多去探索优化的
方案。
第四,当实验结果与预期不符,需要从多个角度去思考原因,并再次调试来验证。
最后是,在做实验遇到阻碍时不要一直埋头苦做,可以适当地寻求帮助,不仅提高效率也可从他人处学到更好的调节方法等。


\end{document}





